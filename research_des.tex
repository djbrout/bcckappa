\documentclass[12pt,preprint]{aastex}
%\documentclass{article}
\usepackage[margin=1.5cm]{geometry}


\pdfoutput=1
\usepackage{epstopdf}
\usepackage{amsmath}
\usepackage{comment} 
\usepackage{appendix}
\usepackage{subfigure}

\usepackage{color}
\newcommand{\xScale}{1.1}   % 1.1 for emulateapj
\newcommand{\xxScale}{1.15}  % 1.15 for emulateapj

\begin{document}
\title{Dan Scolnic - Research Statement - IAU Fellowship}

\textbf{The Next Generation of Dark Energy Constraints from Type Ia Supernovae}

Understanding the nature of dark energy is widely considered one of the greatest challenges in cosmology today.  Dark energy comprises roughly $70\%$ of the universe, yet there is no complete explanation for its existence.  Dark energy was first discovered fifteen years ago (\citealp{Riess98}, \citealp{Saul99}) when observations of high redshift Type Ia Supernova (SNIa) revealed evidence of an accelerated rate of cosmic expansion due to an apparently repulsive-like gravity.  SN\,Ia are still one of the best cosmological probes because they all have similar physical underpinnings and they can be discovered in large sample sizes.  Because of this, there has been a concerted effort to probe the systematic uncertainties in the current samples and to discover increasingly larger samples of SN\,Ia.  I have been fortunate to lead projects to address both of these efforts.  I wish to continue this work at the Institute of Cosmology and Gravitation at Portsmouth.

As an IAU fellow at ICG Portsmouth, I propose to work with scientists to use the Dark Energy Suvey (DES) supernova sample to determine the tightest constraints on dark energy.  I also propose to work on a new project to solve the mystery of supernova color, one of the largest systematics in the current supernovae analyses and likely the largest in the next DES analysis.  I have helped show how important the Pan-STARRs sample is to determine the nature of dark energy, and feel I am in a unique position to optimize the strengths of the DES sample.  I have strong collaborations with scientists at ICG Portsmouth, and would love the opportunity to work with them on this extremely important endeavor.

\textbf{PS1 and DES Constraints of Dark Energy}


One of the most exciting developments in the pursuit to understand dark energy is that there are now multiple experiments that are able to measure dark energy in very different ways (e.g., measurements of the cosmic microwave background or measurements of baryon acoustic oscillations).  In our analysis of the PS1 sample, we found that the measurements of dark energy from these different experiments, including our own, are in tension with the simplest explanation of dark energy.  We found that the equation-of-state of dark energy parameter $w=-1.186^{+0.076}_{-0.065}$ when combining PS1+\textit{Planck}+BAO+$H_0$.  This result is $2.5\sigma$ from $w=-1$, the value that describes the `cosmological constant'.  There are different explanations for this tension: one, that some of the experiments are incorrect, or two, that there is either new physics to be discovered or dark energy is best described by a dynamical model. It is now imperative that the precision and accuracy of each probe is improved and I show in Table 1 cosmological parameters from the different measurements (Scolnic et al. 2013b).  This is a fascinating time as so many very different subfields of cosmology have intersected to answer a very important question.  In the PS1 papers, we brought to light  the significance of the tension between the various experiments, and I hope to be involved in understanding its source.  Bob Nichol has built a fantastic SN group at Portsmouth with a lot of experience in tackling the major issues that such a large survey will face.  Together I feel we can take the next step in better understanding cosmology.


\begin{table}[ht]
\caption{Dependency of cosmological constraints on samples}
\begin{tabular}{l|c|c|c|c|c}
\hline \hline
Probe & \textit{Planck} & \textit{WMAP} &\textit{Planck}+PS1-lz &\textit{WMAP}+PS1-lz   \\
~ & $\Omega_M~~~~w$  & $\Omega_M~~~~w$  & $\Omega_M~~~~w$&  $\Omega_M~~~~w$} \\
\hline
- & $0.22^{0.02}_{-0.08}$ ~~~~ $-1.49^{0.25}_{-0.43}$ & $0.30^{0.07}_{-0.16}$ ~~~~ $-1.02^{0.67}_{-0.26}$ &  $ 0.27^{+0.02}_{-0.02}$  ~~~~  $-1.17^{+0.08}_{-0.08}$  &  $ 0.26^{+0.02}_{-0.03}$  ~~~  $-1.11^{+0.09}_{-0.08}$ \\
+BAO & $0.29^{0.02}_{-0.02}$ ~~~~ $-1.13^{0.14}_{-0.10}$ & $0.30^{0.02}_{-0.02}$ ~~~~ $-0.98^{0.17}_{-0.11}$ &  $ 0.28^{+0.01}_{-0.01}$  ~~~~  $-1.15^{+0.08}_{-0.07}$ &  $ 0.28^{+0.01}_{-0.02}$  ~~~  $-1.10^{+0.09}_{-0.14}$ \\
+$H_0$ & $0.26^{0.02}_{-0.02}$ ~~~~ $-1.24^{0.09}_{-0.09}$ & $0.25^{0.02}_{-0.02}$ ~~~~ $-1.14^{0.12}_{-0.10}$ &  $ 0.26^{+0.01}_{-0.02}$  ~~~~  $-1.19^{+0.07}_{-0.07}$ &  $ 0.25^{+0.01}_{-0.02}$  ~~~  $-1.12^{+0.08}_{-0.07}$ \\
+Union2 & $0.29^{0.02}_{-0.03}$ ~~~~ $-1.09^{0.09}_{-0.08}$ & $0.28^{0.02}_{-0.03}$ ~~~~ $-1.02^{0.09}_{-0.09}$ &  $ 0.28^{+0.01}_{-0.02}$  ~~~~  $-1.13^{+0.06}_{-0.06}$ &  $ 0.27^{+0.02}_{-0.02}$  ~~~  $-1.07^{+0.07}_{-0.07}$ \\
+SNLS & $0.28^{0.02}_{-0.02}$ ~~~~ $-1.13^{0.07}_{-0.07}$ & $0.26^{0.02}_{-0.02}$ ~~~~ $-1.07^{0.07}_{-0.07}$ &  $ 0.28^{+0.01}_{-0.02}$  ~~~~  $-1.14^{+0.06}_{-0.05}$  &  $ 0.26^{+0.02}_{-0.02}$  ~~~  $-1.08^{+0.06}_{-0.06}$ \\
+PS1-lz & $ 0.27^{+0.02}_{-0.02}$  ~~~~  $-1.17^{+0.08}_{-0.08}$  &  $ 0.26^{+0.02}_{-0.03}$  ~~~  $-1.11^{+0.09}_{-0.08}$ & - & -\\
\hline
\end{tabular}
Constraints on $\Omega_M$ and $w$ for different combinations of cosmological probes.  We assume zero curvature.  Columns 2-5 show constraints when only CMB measurements are analyzed with the measurements stated in the left-most column.  Columns 6-9 show constraints from when measurements of the CMB and PS1-lz are combined with the other probes.
\end{table}

Since completing the analysis of the first PS1 sample, I am now working on analysis of the full sample of SNIa discovered by PS1 over the entire four years of the survey.  The PS1 sample analyzed by Rest et al. and Scolnic et al. for cosmology had 112 SNe and the full sample has $>300$ spectroscopically confirmed SNIa.  We can therefore reduce our statistical uncertainties by $\sim \sqrt(3)$, and I plan to also lead the effort to combine our sample with that of SDSS and SNLS.  The combination of these three large samples will involve significant collaboration with many members of the SDSS SNe team to combine the SDSS and SNLS samples.  This sample of $>300$ PS1 SNIa is still only a fraction on the total number of SNIa discovered by the PS1 survey.  I have worked on new methods to take advantage of a photometric SN sample (Scolnic et al. 2009, Scolnic et al. 2013c).  I would be able to take the PS1 dataset and analysis with me to ICG Portsmouth which could be an indispensable asset to the Dark Energy SN survey analysis.  Many of the issues that DES faces and will face are shared by PS1 as both are largely photometric SN samples.  Since we are now entering this new generation of SN analyses in which we must rely on SN classification, there are new systematic uncertainties to analyze.  The PS1 sample can be a critical testbed for the DES analysis.  
 
 In my paper on systematic uncertainties in the PS1 measurements, I showed that the magnitude of our systematic uncertainties may soon surpass the magnitude of our statistical uncertainties.  There are a number of these systematic uncertainties that that will be lower with the DES survey due to its amazing design, but a number still that need much more work to reduce.  Calibration is the largest PS1 uncertainty, and observations of Calspec standards (to tie to absolute calibration), measurements of atmospheric extinction, and measurements of the filter transmission across the filter plane will be much better with DES.  Since I have seen how limiting these uncertainties can be, I have a great appreciation for the planning DES has done.  I have put significant work into reducing uncertainty in the Milky Way extinction and in how to deal with spectroscopic selection effects.  Furthermore, I would like to work on a specific program relating to one of the largest uncertainties in the PS1 analysis and future DES analysis: supernova color.


\begin{comment}
\begin{figure}
\centering
\epsscale{.5}  % 1.15 for emulateapj
\plotone{MSFigures/planckw.pdf}
\caption{The distribution for SNLS3 simulations with a conventional luminosity-variation model (top) and the `Milky Way' color-variation model.  The input ($c_{\textrm{mod}}$) distribution and observed distribution are shown for each simulation, as well as the true SNLS3 observed color distribution.  The parameters of the input distribution for each simulation are given.}
\label{fig:MLCS}
\end{figure}
\end{comment}


   

 


\textbf{The Mystery of Blue Type Ia Supernovae}

One of the dominant systematic uncertainties in measurements of SNIa distances is due to incomplete understanding of SN\,Ia physics.  In particular, we don't understand how the color of a SN\,Ia is related to its luminosity.  Previous analyses had empirically shown that the difference in color, SN to SN, does not match the difference in brightness as one would anticipate from Milky Way Reddening.  In Scolnic et al. (2013) I showed this conclusion changes if one accounts for intrinsic variation in SN Ia colors.  In the systematics paper, I allow for either possibility: that SN color is consistent with MW reddening and can be explained as part intrinsic color and part reddening, or that SN color is not consistent with MW reddening and there is no difference between the various components of SN color.  Since I allow for either possibility, this uncertainty was one of the largest in our analysis ($\sigma_w$ up to $6\%$).  To break the degeneracy between the two different understandings of SN color will take a devoted research program.  While for the PS1 analysis, the uncertainty due to calibration was still the dominant systematic uncertainty, with the improved calibration of DECam, the color uncertainty will likely be the biggest uncertainty.  


\begin{figure}[h]
\epsscale{1.05}
\plottwo{ScatterFigures/MLCSc.pdf}{ScatterFigures/hatfield.pdf}
%The SNLS candidates NU, OE and JH are plotted with their photometric measurements using the SNACC filters, as well as their spectrophotometric positions. (Right) Design of Four Tooth Filters, including the PanSTARRS g' and r' as the broad filter. 
\caption{(Left) The observed color distribution for SNLS3 simulations with the conventional luminosity-variation model (top) and the `Milky Way' color-variation model (bottom).  The input color distribution (blue) and predicted distribution (red) are shown for each simulation, as well as the true SNLS3 observed color distribution (black).  The parameters of the input distribution for each simulation are given. (Right) The relation between Hubble residuals and color for the full SNLS3+SDSS+Nearby sample, a simulated sample based on Milky Way-like extinction, and the conventional empirical model.  The slopes of the trend of Hubble residuals with colors for both blue ($c<0$) and red ($c>0$) colors are shown.  Figures from Scolnic et al. 2013a.   }
\end{figure}

In Fig. 1 I show analysis from Scolnic 2013a that explains how we may be able to break the degeneracy of the different models for SN colors.  On the left, I show the observed color distributions of the SNLS3 sample as well as what we can predict from accurate simulations of these samples.  On the top, I model color using the conventional assumption that the intrinsic scatter seen in SNIa distances does not have a significant color component and that there is no difference between a reddening component of the color and any intrinsic color of the SNe.  On the bottom, I show how if we assume that the color has a component that is due to MW reddening (the color $c>-0.1$), and that the intrinsic color is largely due to color variation, then we can predict the empirical color-luminosity relation that has been long believed to show MW reddening is not the correct model.  In the figure on the right, I show that there is a subtle way to break this degeneracy and that is to see if blue supernovae have the same relation with luminosity as the red supernovae.  Excitingly, we see a significant discrepancy between blue and red colors in the data. 

To further improve this understanding of SN color, we must learn more about the blue ($c<-0.1$), reddening-free, supernovae where there is the most leverage on this discrepancy.  What is particularly interesting is that in the entire low-z sample, $<1\%$ of the SNe have colors which are significantly blue ($c<-0.15$).  However, $\sim 15\%$ of the SNe in the higher-z samples are significantly blue.  These blue SNe have a strong pull on the cosmological parameters recovered.  The question then is if we've rarely seen a `blue' SN at low-z, do `blue' SNe really exist?  They may reveal different selection effects between the high and low-z samples (which is very degenerate with cosmology), or may reflect that our light curve fitters need more informed priors (similarly problematic for cosmology).  Trying to answer this fairly simple question of whether or not blue SNe exist is likely the most direct path towards understanding both the reddening and intrinsic color of SNe.   I want to begin a project at ICG Portsmouth to resolve this discrepancy.  

This research program can largely be possible with spectroscopic follow-up plans already in place, though will likely need further telescope resources to reach the necessary statistics.  The DES supernova survey gets about ten hours of spectroscopic large-field followup at AAT a year.  Most of the planning for the follow-up of the survey is done at ICG Portsmouth led by Chris D'Andrea.  This will allow nearly complete host-galaxy follow-up of the supernovae, but also a small fraction of the spectra of the supernovae themselves.    Since the photometric uncertainty of the SNe light curves will be so small with DECam at redshifts near $z\sim0.5$, any intrinsic scatter/color of the supernovae can be well separated from photometric uncertainty.  Separately, I plan to apply for spectroscopic time at a large telescope for high-resolution follow-up of any blue DES SNIa which we would be able to classify early.  Good spectral information will be really important to better probe the intrinsic color.  Using the blue supernovae, or determining if they exist, will allow us to finally tackle the mystery of SN color.




\textbf{Conclusion}

There is amazing research currently going on at ICG Portsmouth, and so much of it is crucial to improve our understanding of dark energy.  DECam is truly a next-generation instrument, but it will take significant work to capitalize on the amazing science that will soon be possible.  I am sure that at ICG Portsmouth we can make great progress in uncovering one of the greatest challenges in science - determining the nature of dark energy.

\bibitem[{Perlmutter {et~al.}(1999)}]{Saul99}
Perlmutter, S. {et~al.} 1999, \apj, 517, 565

\bibitem[{Riess {et~al.}(1998)}]{Riess98}
Riess, A. {et~al.} 1998, \aj, 116, 1009

\bibitem[{Riess {et~al.}(1998)}]{Scolnic}
Scolnic, D. {et~al.} 2009, \apj, 706, 94

\bibitem[{Riess {et~al.}(1998)}]{Scolnic}
Scolnic, D. {et~al.} 2013a, accepted by ApJ, arXiv.org:1306.4050  

\bibitem[{Riess {et~al.}(1998)}]{Scolnic}
Scolnic, D. {et~al.} 2013b, submitted to ApJ, arXiv.org:1310.3824 

\bibitem[{Riess {et~al.}(1998)}]{Scolnic}
Scolnic, D. {et~al.} 2013c, in prep.

\bibitem[{Riess {et~al.}(1998)}]{Rest}
Rest, A. {et~al.}  2013, arXiv.org:1310.3828 






\end{document}